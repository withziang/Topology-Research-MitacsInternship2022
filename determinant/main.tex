\documentclass[12pt,letterpaper]{article}
\usepackage[utf8]{inputenc}
\usepackage{amsmath, amsthm, amssymb}
\newtheorem{mydef}{Definition}
\newtheorem{prop}{Proposition}

\begin{document}
\begin{center}
    \huge\textbf{Determinant} \\[10pt]
    \large\textit{Z.W}  \\[20pt]
\end{center}
\section{Permutation}
\begin{mydef}\label{def:def444}
Let $S$ be a set and $\{S\}=\{a_1,a_2,...,a_n\}$, the permutation of this set is a reordering of this set. It is quite obvious to see that permutation of a set is not unique. One permutation of $S$ can be mathematically represented as 
$$\sigma : \{a_1,a_2,...,a_n\} \longrightarrow \{\sigma(a_1),\sigma(a_1),...,\sigma(a_n)\}$$
where  $\sigma(a_1)$ is any number in the set $S$, and when $i \geqslant 1$, $\sigma(a_i)$ is any element in the set $\{S \setminus(\bigcup_{k=1}^{i-1}\sigma(a_i))\}$.
\\
\newline Permutation of $S$ can also be represented by Cauchy's two line notation,
$$\sigma= \large(\begin{smallmatrix}
    a_1 & a_2 & a_3 & \cdots & c_{n-1} & c_n \\
    \sigma(a_1) & \sigma(a_2) & \sigma(a_3) & \cdots &  \sigma(a_{n-1})  & \sigma(a_n)
  \end{smallmatrix}\large)$$
 Note: when it is $\sigma\{a,b,c\}$, $a,b,c$ denote the position of the permutation.\\[10pt]
\end{mydef}

\begin{mydef}\label{def:def444}
Let $S$ be a set and $\{S\}=\{a_1,a_2,...,a_n\}$. Let $\sigma$ be a permutation of $S$. This permutation can be presented in the form of a matrix which is denoted by $[A_\sigma]_{n*n}$.
\begin{equation}
 \begin{cases}
 (A_\sigma)_{ij}=1,\textbf{     } \textbf{  }\textbf{  }\textbf{  }if \textbf{  }i=\sigma(j). \\
 (A_\sigma)_{ij}=0,\textbf{     } \textbf{  }\textbf{  }\textbf{  } if\textbf{  } i\neq \sigma(j).
 \end{cases} 
\end{equation}
To have an intuition of this equation, a permutation matrix has only one emtry that is 1 in each column and row, the rest of the entries are just 0.
\end{mydef}
\begin{mydef}\label{def:def444}
A transposition is a permutation that only interchange two elements and leave the rest fixed. For which, there are $i,j$ that
\begin{equation}
 \begin{cases}
 \sigma(i)=j, \\
 \sigma(j)=i, \\
 \simga(k)=k, \hspace{1em} for \textbf{ } every  \textbf{ } k \neq i,j 
 \end{cases} 
\end{equation}

\begin{prop}
Every permutation can be written as a finite composition of transpositions. 
\end{prop}
Proof: Consider the reverse of the process. \\
Let the set $S=\{a_1,a_2,...,a_n\}$ \\
Denote  $$b_1=min\{a_1,a_2,...,a_n\}$$
$$\cdots$$
$$b_i=min(S\setminus(\bigcup_{k=1}^{i-1}\{b_k\}))$$
$$\cdots$$
$$b_n=min(S\setminus\{b_1,b_2,...,b_{n-1}\})$$
Step 1 of the swiping process: transposition between $a_1$ and $b_1$, so $S$ becomes $$\{b_1,...,a_1,...,a_n\}, \hspace{2em} \text{if } b_1 \neq a_1$$
or be $$\{b_1,a_2,...,a_n\}, \hspace{3em} \text{if } b_1 = a_1$$
Do this process for n times, transposition between $a_i$ and $b_i$ and this is always well defined since $a_i$ and $b_i$ always exist in the set (axiom of choice).After the process, $S$ becomes $\{b1,b_2,...,b_n\}$. So any set can become an increasing order set only using transposition. The reverse of this process is also a composition of transposition also.\\
$$\text{any set} \longrightarrow \text{increasing order permutation} 
\longrightarrow \text{any set}$$
So proven.\\
Note: Transposition can be considered as an elementary permutation. If we denote each transposition as $\pi$, then every permutation can be written as $\sigma =\pi_1 \circ \pi_2 \circ ..... \circ \pi_k$. 
\end{mydef}
\begin{mydef}\label{def:def444}
Let $S$ be a set and $\{S\}=\{a_1,a_2,...,a_n\}$. Let $\sigma$ be a permutation of $S$. The signature of this permutation is defined as 
$$\varepsilon(\sigma)=\frac{\Pi_{1 \leq i<j \leq n }(x^{\sigma(i)}-x^{\sigma(j)})}{\Pi_{1 \leq i<j \leq n }(x^{i}-x^{j})}.$$
This function sends set $\sigma(S)$ to $\{-1,1\}$
\end{mydef}
\newpage
\begin{prop}
If $\sigma$ is a transposition, $\varepsilon(\sigma)=-1$. \\
Proof: \\
Let $S$ be a set and $\{S\}=\{a_1,a_2,...,a_n\}$. Let $\sigma$ be a transposition between the element $a_p$ and $a_q$ for $p,q \in {1,2,3...,n}$. Without the lost of generality, we presume $p<q$.  \\
So $$\varepsilon(\sigma)=\frac{
[\Pi_{1 \leq i<j \leq n,}(x^{\sigma(i)}-x^{\sigma(j)})][\pi_{}]}
{\Pi_{1 \leq i<j \leq n }(x^{i}-x^{j})}$$


\end{prop}

\begin{prop}
For two transposition permutation $\sigma_1$ and $\sigma_2$,
$$\varepsilon(\sigma_1 \circ \sigma_2)=\varepsilon(\sigma_1) \cdot \varepsilon(\sigma_2) $$ 
Proof: \\

\end{prop}

\end{document}